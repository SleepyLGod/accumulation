% hw10.tex 
% Created by Jack Kasbeer (jkasbeer) on November 27, 2015
%
% Principles of Functional Programming
% 15-150
% Assignment 10
\documentclass[11pt]{article}

\usepackage{amsmath}
\usepackage{amsthm}
\usepackage{amssymb}
\usepackage{fancyhdr}

\oddsidemargin0cm
\topmargin-2cm     
\textwidth16.5cm   
\textheight23.5cm  

\newcommand{\question}[2] {\vspace{.25in} \hrule\vspace{0.5em}
\noindent{\bf #1: #2} \vspace{0.5em}
\hrule \vspace{.10in}}
\renewcommand{\part}[1] {\vspace{.10in} {\bf (#1)}}

\newcommand{\myname}{Jack Kasbeer}
\newcommand{\myandrew}{jkasbeer@andrew.cmu.edu}
\newcommand{\myhwnum}{10}

\setlength{\parindent}{0pt}
\setlength{\parskip}{5pt plus 1pt}
 
\pagestyle{fancyplain}
\lhead{\fancyplain{}{\textbf{HW\myhwnum}}}     
\rhead{\fancyplain{}{\myname\\ \myandrew}}
\chead{\fancyplain{}{15-150}}

\begin{document}

\medskip            
\thispagestyle{plain}
\begin{center} 
{\Large 15-150 Assignment \myhwnum} \\
\myname \\
\myandrew \\
Section K \\
November 17, 2015 \\
\end{center}

% ##### TASK 2 ##### %
\question{2}{Tree Sequences}
\begin{enumerate}
% Task 2.1 
\item $W_{length}(n) = c_0 + c_1 + 2*W_{length}(n div 2)$\\
$\Rightarrow W_{length}(n) = O(n)$\\
$S_{length}(n) = c_0 + c_1 + 2*W_{length}(n\,div\,2)$\\
$\Rightarrow S_{length}(n) = c_2 + W_{length}(n\,div\,2)$ (parallelism)\\
$\Rightarrow S_{length}(n) = O(log\,n)$
% Task 2.2 
\item \verb|nth| is implemented in \verb|shrubseq.sml|.
% Task 2.3
\item \verb|tabulate| is implemented in \verb|shrubseq.sml|.
% Task 2.4
\item \verb|length| is implemented in \verb|sizeseq.sml|.
% Task 2.5
\item \verb|nth| is re-implemented in \verb|sizeseq.sml|.
% Task 2.6
\item \verb|tabulate| is implemented in \verb|sizeseq.sml|.
% Task 2.7
\item The work and span for both of these implementations for \verb|tabulate| is the same.  The difference between them is the cost trade-off; in \verb|sizeseq.sml|, the \verb|tabulate| function uses more memory because of the extra mode.  The \verb|length| function differs in the sense that it's not recursive in \verb|shrubseq.sml|, and as a result its work is $O(n)$ in \verb|shrub|, but $O(1)$ in \verb|size|.  The span is the same for both due to parallelism.
\end{enumerate}

% ##### TASK 3 ##### %
\question{3}{Just a Monoid in the Category of Endofunctors}
\begin{enumerate}
% Task 3.1 
\item \verb|id = (fn x => SOME x)|
% Task 3.2 
\item \verb|(fn 0 => NONE| $|$ \verb|x => SOME ((3 mod x) + 1)|
% Task 3.3 
\item \verb|findN| is implemented in \verb|findMany.sml|
\end{enumerate}

% ##### TASK 4 ##### %
\question{4}{Barnes-Hut}
\begin{enumerate}
% Task 4.1
\item \verb|barycenter| is implemented in \verb|barnes-hut.sml|
% Task 4.2
\item \verb|quadrantize| is implemented in \verb|barnes-hut.sml|
% Task 4.3
\item \verb|compute_tree| is implemented in \verb|barnes-hut.sml|
% Task 4.4
\item \verb|groupable| is implemented in \verb|barnes-hut.sml|
% Task 4.5
\item \verb|bh_acceleration| is implemented in \verb|barnes-hut.sml|
\end{enumerate}

\end{document}


















