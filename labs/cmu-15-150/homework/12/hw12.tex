% hw12.tex 
% Created by Jack Kasbeer (jkasbeer) on December 7, 2015
%
% Principles of Functional Programming
% 15-150
% Assignment 12
\documentclass[11pt]{article}

\usepackage{amsmath}
\usepackage{amsthm}
\usepackage{amssymb}
\usepackage{fancyhdr}

\oddsidemargin0cm
\topmargin-2cm     
\textwidth16.5cm   
\textheight23.5cm  

\newcommand{\question}[2] {\vspace{.25in} \hrule\vspace{0.5em}
\noindent{\bf #1: #2} \vspace{0.5em}
\hrule \vspace{.10in}}
\renewcommand{\part}[1] {\vspace{.10in} {\bf (#1)}}

\newcommand{\myname}{Jack Kasbeer}
\newcommand{\myandrew}{jkasbeer@andrew.cmu.edu}
\newcommand{\myhwnum}{12}

\setlength{\parindent}{0pt}
\setlength{\parskip}{5pt plus 1pt}
 
\pagestyle{fancyplain}
\lhead{\fancyplain{}{\textbf{HW\myhwnum}}}     
\rhead{\fancyplain{}{\myname\\ \myandrew}}
\chead{\fancyplain{}{15-150}}

\begin{document}

\medskip            
\thispagestyle{plain}
\begin{center} 
{\Large 15-150 Assignment \myhwnum} \\
\myname \\
\myandrew \\
Section K \\
December 8, 2015 \\
\end{center}

% ##### QUESTION 4 ##### %
\question{4.2}{RegExTest with Strings}
Results:\\
*******************************\\
Tests for Regular:\\
Test R0: ['', a, aa, aaa, aaaa, aaaaa, aaaaaa, aaaaaaa, aaaaaaaa, aaaaaaaaa]\\
Test R1: ['', a, b, aa, ab, bb, aaa, aab, abb, bbb]\\
Test R2: ['', a, b, aa, bb, aaa, bbb, aaaa, bbbb, aaaaa]\\
Test R3: ['', '', '', '', '', '', '', '', '', '']\\
Test R4: ['', a, b, aa, ab, ba, bb, aaa, aab, aba]\\
Test R5: ['', ab, abab, ababab, abababab, ababababab, abababababab, ababababababab, abababababababab, ababababababababab]\\
Test R6: ['']\\
Test R7: ['', '', '', '', '', '', '', '', '', '']\\
Test R8: ['', '', '', '', '', '', '', '', '', '']\\
Test R9: ['', ab, abab, ababab, abababab, ababababab, abababababab, ababababababab, abababababababab, ababababababababab]\\
Test R10: [abc]\\
Test R11: [abc]\\
Test R12: [a, b, c]\\
Test R13: [a, b, c]\\
Test R14: [aaa]\\
Test R15: [aaa]\\
Test R16: [a]\\
Test R17: [a]\\

\newpage

\question{4.3}{RegExTest with Strings2}
Results:\\
*******************************\\
Tests for Regular2:\\
Test R0: ['', a, aa, aaa, aaaa, aaaaa, aaaaaa, aaaaaaa, aaaaaaaa, aaaaaaaaa]\\
Test R1: ['', a, aa, aaa, aaaa, aaaaa, aaaaaa, aaaaaaa, aaaaaaaa, aaaaaaaaa]\\
Test R2: ['', a, aa, aaa, aaaa, aaaaa, aaaaaa, aaaaaaa, aaaaaaaa, aaaaaaaaa]\\
Test R3: ['', '', '', '', '', '', '', '', '', '']\\
Test R4: ['', a, aa, aaa, aaaa, aaaaa, aaaaaa, aaaaaaa, aaaaaaaa, aaaaaaaaa]\\
Test R5: ['', ab, abab, ababab, abababab, ababababab, abababababab, ababababababab, abababababababab, ababababababababab]\\
Test R6: ['']\\
Test R7: ['', '', '', '', '', '', '', '', '', '']\\
Test R8: ['', '', '', '', '', '', '', '', '', '']\\
Test R9: ['', ab, abab, ababab, abababab, ababababab, abababababab, ababababababab, abababababababab, ababababababababab]\\
Test R10: [abc]\\
Test R11: [abc]\\
Test R12: [a, b, c]\\
Test R13: [a, b, c]\\
Test R14: [aaa]\\
Test R15: [aaa]\\
Test R16: [a]\\
Test R17: [a]\\

The results differ because in \verb|Strings|, the comparison function cases on the length of each element before comparing their dictionary order.  In contrast, \verb|Strings2| uses the typical \verb|String.compare| and nothing else to compare the elements.  In the case of \verb|Test R2|, this causes "b" to come before "aa" in \verb|Regular|, even though "aa" has a dictionary value smaller than "b" (which is why strings of a's of increasing length are the first ten elements in \verb|Regular2| for \verb|Test R2|).   The effects of this difference can be seen specifically in \verb|Test R1|, \verb|Test R2|, and \verb|Test R4| (only a few examples).

% ##### QUESTION 5 ##### %
\question{5.2}{MemoKnuth results}
\verb|structure S = Memo_Knuth(Table)(SimpleNim)|

\verb|S.player 100 = 3|

\end{document}


















